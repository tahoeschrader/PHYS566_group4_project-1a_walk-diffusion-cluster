%%% Group 4: Tahoe, Ksenia Sokolova, Xinmeng Tong
%%% Computational Physics
%%% Spring, 2017

%%%%%%%%%%%%%%%%%%%%%%%%%%%%%%%%%%%%%%%%%%%%%%%%%%%%%%%%%%%%%%%%%%%%%%%%%%%%%%%%
%%% This code will develop a LaTeX'd writeup of the second group project
%%% assignment for PHYS566.
%%%%%%%%%%%%%%%%%%%%%%%%%%%%%%%%%%%%%%%%%%%%%%%%%%%%%%%%%%%%%%%%%%%%%%%%%%%%%%%%

% PACKAGES AND OTHER DOCUMENT CONFIGURATIONS
\documentclass[12pt]{article}
\usepackage[english]{babel}
\usepackage[utf8]{inputenc}
\usepackage{amsmath,amsfonts,amssymb}
\usepackage{graphicx,xcolor}
\usepackage{subfig}
\usepackage{booktabs,hyperref}
\usepackage[left=2cm,%
right=2cm,%
top=2cm,%
bottom=2cm,%
headheight=11pt,%
letterpaper]{geometry}%
\usepackage{fancyhdr}
\pagestyle{fancy}
\lhead{\small\sffamily\bfseries\leftmark}%
\chead{}%
\rhead{\small\sffamily\bfseries\rightmark}
\renewcommand{\headrulewidth}{1pt}
\renewcommand{\footrulewidth}{1pt}
\graphicspath{{images/}}

% Article Information
\title{Percolation of a 2D Lattice}
\author{Tahoe Schrader, Ksenia Sokolova, Xinmeng Tong \\PHYS566}
\date{}

% Begin writing document
\begin{document}
\maketitle

% Start with the abstract

\abstract{In this assignment we computationally explore various $2D$ lattices and generate their spanning clusters, $p_c$ and $\beta$. These variables refer to the probability a spanning cluster arises at and the critical exponent of a spanning cluster, respectively. Theoretically, these values for an infinite $2D$ lattice are $p_c\approx 0.593$ and $\beta=\frac{5}{36}$. \\
Our code is available on GitHub: \\

\url{www.github.com/tahoeschrader/PHYS566_group4_projects}}

\section{Theory}
\label{sec:theory}
Percolation theory describes the behavior of connected clusters. This theory has many important relations to physics and chemistry. For example, it can be used to describe the movement and filtering of fluids through porous materials. When a system \emph{percolates}, a $2^\text{nd}$ order phase transition can also result.

The act of percolation refers to the existence of a spanning cluster. A spanning cluster is a cluster that reaches all edges of a physical boundary. For example, a spanning cluster must touch all four sides of a $2D$ box. Percolation, i.e. a system exhibiting a spanning cluster, is interesting because universality is \emph{also} exhibited. Universality is when the properties of the system are independent of the systems dynamical details.

One feature of the percolating system that we would like to document is the clustering probability, $p_c$, where generally
\begin{equation}
  \label{eq:probability}
  p = \frac{\text{\# of occupied lattice sites}}{\text{total lattice sites}}.
\end{equation}

Probability ranges and the theoretical results of such values are displayed in Table~\ref{table:pvalues}.
\begin{table}[!htb]
\centering{\minipage{0.75\textwidth}
  \centering{\begin{tabular}{ l | l }
    \hline
    small $p$ &  isolated clusters \\
    $p\approx 0.4$ &  many small clusters \\
    $p\approx 0.6$ &  large, barely connected clusters \\
    $p\approx 0.8$ &  most sites belong to same cluster \\
    \hline
  \end{tabular}}
  \caption{The physical representation of various occupation probabilities on a lattice.}
  \label{table:pvalues}
\endminipage}
\end{table}

Another important value in percolation theory is the fraction of all sites in the spanning cluster with respect to all occupied sites
\begin{equation}
  \label{eq:fraction}
  F = \frac{\text{\# of spanning sites}}{\text{total \# of cluster sites}}.
  \end{equation}
This value also takes the form
\begin{equation}
  \label{eq:spanningfraction}
  F = F_0 \left(p - p_c\right)^\beta,
\end{equation}
where $\beta$ is the critical exponent.

In the infinite size limit, this fraction parameter is proved to have the singular behavior
\begin{equation}
  \label{eq:spanningfraction2}
F(p) =\left\{
\begin{aligned}
&0 \ \ & p<p_c \\
&F_0(p-p_c)^\beta \ \ & p\geq p_c 
\end{aligned}
\right.
\end{equation}
,where the order parameter critical exponent $\beta$ = 5/36 for percolation on a 2-D square lattice.


\subsection{2D Lattice Systems}
\label{sec:lattice}

The steps for generating a $2D$ percolating lattice are quite simple:
\begin{enumerate}
  \label{code:percolatinglattice}
  \item Generate a 2D lattice
  \item Populate a single lattice site at random and define it to be a cluster
  \item Populate another single lattice site at random
  \begin{enumerate}
    \item If the new site touches an old site, add it to the cluster
    \item If the new site doesn't touch an old site, define it to be a new cluster
    \item If the new site touches multiple different clusters, choose the smaller numbered cluster to "win" and overtake the other clusters
  \end{enumerate}
  \item Repeat until a spanning cluster arises
\end{enumerate}

It is evident that larger values of $p$ will more likely result in the existence of a spanning cluster than smaller values of $p$. Interestingly, the transmission from many clusters to a spanning cluster is sharp, qualifying it as a phase transition. For an infinite $2D$ lattice, the theoretical probability where this occurs is
\begin{equation}
  \label{eq:theoreticalpc}
  p_c \approx 0.593
\end{equation}
whereas the critical exponent of the spanning cluster fraction is
\begin{equation}
  \label{eq:theoreticalbeta}
  \beta = \frac{5}{36}.
\end{equation}

In Section~\ref{sec:results}, we computationally explore such a $2D$ lattice and generate spanning clusters, $p_c$ and $\beta$ for various lattice sizes.

\section{Results}
\label{sec:results}

\subsection{Extracting $p_c$}
\label{sec:pc}

First we determine the critical probability by running the code for $N=5, 10, 15, 20, 30, 50, 80$ for 50 times, every time finding the $p_c$.

To do this computationally, we used the cluster labeling method outlined above. Before running the averaging step, we tested the performance of our code by checking the dependencies between the time required to run one iteration and the size of the matrix. Please see results on Figure ~\ref{fig:timeElapsed}. We observe that our code is complexity $O[N^2]$, this is expected from the algorithm. This means that the larger the matrix, the worse the code performs, with squared scaling. And the change for the bad side happens relatively quickly. While this is durable for the cases we are observing, for larger matrices we would need to switch to the different algorithm.
\begin{figure}[!htb]
\centerline{\minipage{.6\textwidth}
  \includegraphics[width=\linewidth]{timeElapsed1.png}
  \caption{Time elapsed and size of the matrix}\label{fig:timeElapsed}
\endminipage}
\end{figure}

Before checking finding the $p_c$ we can also visually observe formation of clusters. (Note: on github project you can also find a .gif of the process). On the Figures ~\ref{fig:snapshotN30One} to ~\ref{fig:snapshotN30Spanning} we see the development of the cluster for $N=30$. In the beginning, as expected, we see many small clusters. As $p$ increases we observe formation of larger clusters, but still no spanning cluster. And at the end we see one large spanning cluster, and just a couple of smaller clusters.

\begin{figure}[!htb]
\minipage{0.3\textwidth}
  \includegraphics[width=\linewidth]{snapshotN30One.png}
  \caption{2D percolation start}\label{fig:snapshotN30One}
\endminipage\hfill
\minipage{0.3\textwidth}
  \includegraphics[width=\linewidth]{snapshotN30Two.png}
  \caption{2D percolation intermediate}\label{fig:snapshotN30Two}
\endminipage\hfill
\minipage{0.3\textwidth}
  \includegraphics[width=\linewidth]{snapshotN30Spanning.png}
  \caption{2D percolation spanning cluster}\label{fig:snapshotN30Spanning}
\endminipage\hfill
\end{figure}

To find critical probability, we ran the code for $N=5, 10, 15, 20, 30, 50, 80$ for 50 times, averaging for every N. Please see the resulting plot attached (Figure ~~\ref{fig:criticalProb}). Extrapolating, we found the critical probability to be \textbf{$0.598$}. This is very close to the theoretical value of $0.593$.
\begin{figure}[!htb]
\centerline{\minipage{.7\textwidth}
  \includegraphics[width=\linewidth]{MAINcriticalProb0593.png}
  \caption{Critical probability plot for 2D percolation}\label{fig:criticalProb}
  \endminipage}
\end{figure}

\subsection{Extracting $\beta$}
\label{sec:beta}
Next, we fixed $N=100$ and computationally obtained Equation~\ref{eq:spanningfraction}.


\end{document}
