%%% Group 4: Tahoe, Ksenia Sokolova, Xinmeng Tong
%%% Computational Physics
%%% Spring, 2017

%%%%%%%%%%%%%%%%%%%%%%%%%%%%%%%%%%%%%%%%%%%%%%%%%%%%%%%%%%%%%%%%%%%%%%%%%%%%%%%%
%%% This code will develop a LaTeX'd writeup of the first group project
%%% assignment for PHYS566.
%%%%%%%%%%%%%%%%%%%%%%%%%%%%%%%%%%%%%%%%%%%%%%%%%%%%%%%%%%%%%%%%%%%%%%%%%%%%%%%%

% PACKAGES AND OTHER DOCUMENT CONFIGURATIONS
\documentclass[10pt]{article}
\usepackage[english]{babel}
\usepackage[utf8]{inputenc}
\usepackage{amsmath,amsfonts,amssymb}
\usepackage{graphicx,xcolor}
\usepackage{subfig}
\usepackage{booktabs}
\usepackage[left=2cm,%
right=2cm,%
top=2cm,%
bottom=2cm,%
headheight=11pt,%
letterpaper]{geometry}%
\usepackage{fancyhdr}
\pagestyle{fancy}
\lhead{\small\sffamily\bfseries\leftmark}%
\chead{}%
\rhead{\small\sffamily\bfseries\rightmark}
\renewcommand{\headrulewidth}{1pt}
\renewcommand{\footrulewidth}{1pt}
%\graphicspath{{}}

% Article Information
\title{Random Walks, the Diffusion Equation, and Cluster Growth}
\author{Tahoe Schrader, Ksenia Sokolova, Xinmeng Tong \\PHYS566}
\date{}

% Begin writing document
\begin{document}
\maketitle

% Start with the abstract
\abstract{In this assignment we blah blah blah}

\section{Theory}
\label{sec:theory}

\subsection{Random Walks in 2D}
Mean and squared displacement
Distance from origin

\subsection{Diffusion Equations and the Finite Difference Form}
\label{sec:diffusionequation}
The diffusion equation in $1D$ is written,
\begin{equation}
  \label{eq:diffusioneqn}
  \frac{\partial\rho(x,t)}{\partial t} = D\nabla^2\rho(x,t),
\end{equation}
where $D$ is the diffusion constant. Equation~\ref{eq:diffusioneqn} is turned into an iterable form by noting: $\rho(x,t) = \rho(i\Delta x, n\Delta t) = \rho(i,n)$. This is the finite difference form\footnote{The finite difference form must be used because the diffusion equation is time dependent. Therefore, a relaxation method cannot be used.}.

After using the formal definition of derivatives and algebraically manipulating Equation~\ref{eq:diffusioneqn} in the finite difference form, we get
\begin{equation}
  \label{eq:diffusioneqn-iterable}
  \rho(i,n+1) = \rho(i,n) + \frac{D\Delta t}{\Delta x^2}\left(\rho(i+1,n) + \rho(i-1,n) - 2\rho(i,n)\right),
\end{equation}
where $\Delta t$ and $\Delta x$ are the step sizes in an iteration. This solution requires knowledge of initial conditions. We must assume that the $x$ displacement is known at times prior to and including $t_n = n\Delta t$. Two consecutive steps prior to the first unknown step is sufficient to solve such an equation. Finally, to guarantee stability, the following criterion must be met
\begin{equation}
  \label{eq:stabilitycriterion}
  \Delta t \leq \frac{(\Delta x)^2}{2D}.
\end{equation}

We will use an initial density profile that is sharply peaked around $x=0$, but extends over a few grid sites to resemble a box. This is sufficient for generating the solution to the diffusion equation. Interestingly, after a couple iterations, the box profile will diffuse into a Gaussian normal distribution. The $1D$ Gaussian normal distribution has the form,
\begin{equation}
  \label{eq:gaussiandistribution}
  \rho(x,t) = \frac{1}{\sqrt{2\pi\sigma(t)^2}}\exp\left(-\frac{x^2}{2\sigma(t)^2}\right),
\end{equation}
where $\sigma(t) = \sqrt{2Dt}$.

The spatial expectation value, $\langle x(t)^2\rangle$, of Equation~\ref{eq:gaussiandistribution} is equal to $\sigma(t)^2$. Expectation values are calculated according to the equation
\begin{equation}
  \label{eq:expectationvalue}
  \langle x\rangle = \int_{-\infty}^\infty f(x)xdx,
\end{equation}
where $f(x)$ is the Guassian normal distribution for our purposes. Because we are looking for $\langle x^2\rangle$, Equation~\ref{eq:expectationvalue} becomes
\begin{align}
  \label{eq:spatialexpectationvalueintegral}
  \langle x^2\rangle &= \int_{-\infty}^\infty \frac{1}{\sqrt{2\pi\sigma(t)^2}}\exp\left(-\frac{x^2}{2\sigma(t)^2}\right) x^2 dx \notag \\
  &= 2\int_{0}^\infty \frac{1}{\sqrt{2\pi\sigma(t)^2}}\exp\left(-\frac{x^2}{2\sigma(t)^2}\right) x^2 dx.
\end{align}
The last step can be done because the $x^2$ term makes it an even function being symmetrically integrated about zero. Now, we make a change of variable $(x=\sigma\sqrt{2})$ to obtain, after a couple steps of algebra,
\begin{equation}
  \label{eq:spatialexpectationvalue-simplified}
  \langle x^2\rangle = \frac{4\sigma^2}{\sqrt{\pi}}\int_0^\infty x^2 \exp(-x^2)dx.
\end{equation}
Integrating the above equation by parts yields
\begin{equation}
  \label{eq:variancesolved}
  \langle x^2 \rangle = \sigma^2
\end{equation}
Typically, we call this equation the variance. % I used mathematica for the last step because actually integrating by parts was pretty difficult

\subsection{Cluster Growth with a DLA Model}
\label{sec:clusterDLAmodel}

\section{Computations}
\label{sec:computations}

\subsection{Random Walks in 2D}
\label{sec:computationsRandomWalk}
To code the basic 2D walk, we assume that the probability to step in either direction left, right, up or down is the same. Then, for every step, we generate a random number. This is equivalent to drawing from Uniform (0,1). We assign every move to one of the segments of .25. If the random numbers falls within that value, we move in the assigned direction. The evidence for a successfully coded $2D$ walker program is given in Figure~\ref{fig:20steps} and \ref{fig:1000steps}.

\begin{figure}[!htb]
\minipage{0.5\textwidth}
  \includegraphics[width=\linewidth]{rWalk1.png}
  \caption{blah blah blah}\label{fig:20steps}
\endminipage\hfill
\minipage{0.5\textwidth}
  \includegraphics[width=\linewidth]{rWalk2.png}
  \caption{blah blah blah}\label{fig:1000steps}
\endminipage\hfill
\end{figure}

\subsection{Diffusion of a Box Density Distribution}
\label{sec:diffusion-boxdensity}
Using the arguments in Section~\ref{sec:diffusionequation}, we solve the $1D$ Diffusion Equation over a period of time. Six different snapshots in time were then fit against Equation~\ref{eq:gaussiandistribution} to show a box shaped density will eventually diffuse into a Gaussian normal distribution.

It should be noted that it was important to place this box density far from the edges of the grid. Otherwise, the grid started to act as a makeshift boundary condition that, over time, made the distribution look less Gaussian. The initial distribution of our $1D$ diffusion solver is shown in Figure~\ref{fig:probdensityinit}.
\begin{figure}[!htb]
\centerline{\minipage{0.5\textwidth}
  \includegraphics[width=\linewidth]{probdensityinit.png}
  \caption{initial box-shaped density}\label{fig:probdensityinit}
\endminipage}
\end{figure}

In Figures~\ref{fig:probdensity1}-\ref{fig:probdensity6}, we took snapshots of the $1D$ density equation and fit a Gaussian to the solution. A value of $\sigma$ was then extracted and compared to the analytical solution, $\sigma=\sqrt{2Dt}$. As can be seen, the fit worked exceptionally well for all time snapshots. The edge effects start to ruin our $\sigma$ fit in Figure~\ref{fig:probdensity6}.
\begin{figure}[!htb]
\minipage{0.5\textwidth}
  \includegraphics[width=\linewidth]{probdensityt10.png}
  \caption{}\label{fig:probdensity1}
\endminipage\hfill
\minipage{0.5\textwidth}
  \includegraphics[width=\linewidth]{probdensityt55.png}
  \caption{}\label{fig:probdensity2}
\endminipage\hfill \\
\minipage{0.5\textwidth}
  \includegraphics[width=\linewidth]{probdensityt110.png}
  \caption{}\label{fig:probdensity3}
\endminipage\hfill
\minipage{0.5\textwidth}
  \includegraphics[width=\linewidth]{probdensityt510.png}
  \caption{}\label{fig:probdensity4}
\endminipage\hfill
\minipage{0.5\textwidth}
  \includegraphics[width=\linewidth]{probdensityt1600.png}
  \caption{}\label{fig:probdensity5}
\endminipage\hfill
\minipage{0.5\textwidth}
  \includegraphics[width=\linewidth]{probdensityt2000.png}
  \caption{}\label{fig:probdensity6}
\endminipage\hfill
\end{figure}

\subsection{Growing Clusters and Extracting Fractal Dimensions}
\label{sec:clustersandfractals}

\end{document} % This ends our document

% INCOMPLETE
